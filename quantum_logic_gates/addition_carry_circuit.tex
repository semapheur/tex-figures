\documentclass[tikz, border=5pt]{standalone}
\usepackage{quantikz}

\begin{document}

\begin{quantikz}[wire types={q,n,q,n,q,n,q}]
  \lstick{\ket{x_3}} & \gate{X} \gategroup[7, steps=5, style={inner xsep=0.3cm, inner ysep=0.4cm, yshift=-0.3cm}]{$\hat{U}_c$} \gategroup[7, steps=1, style={draw=none,fill=gray!20, inner ysep=0cm, yshift=-0.1cm},background, label style={label position=below, anchor=north, yshift=-0.2cm}]{$\hat{C}$} & \ \ket{x_1 x_2 \overset{2}{\oplus} x_3}\ & \gategroup[7, steps=1, style={draw=none,fill=gray!20, inner ysep=0cm, yshift=-0.1cm},background, label style={label position=below, anchor=north, yshift=-0.2cm}]{$\hat{D}$} && \gate{X} \gategroup[7, steps=1, style={draw=none,fill=gray!20, inner ysep=0cm, yshift=-0.1cm},background, label style={label position=below, anchor=north, yshift=-0.2cm}]{$\hat{E}$} & \rstick{\ket{x_0 (x_1 \overset{2}{\oplus} x_2) \overset{2}{\oplus} x_1 x_2 \overset{2}{\oplus} x_3}} \\[-0.2cm]
  \lstick{\otimes} &&&&&& \rstick{\otimes} \\[-0.2cm]
  \lstick{\ket{x_2}} & \ctrl{-2} & \ \ket{x_2}\ & \gate{X} & \ \ket{x_1 \overset{2}{\oplus} x_2}\ & \ctrl{-2} & \rstick{\ket{x_1 \overset{2}{\oplus} x_2}} \\[-0.2cm]
  \lstick{\otimes} &&&&&& \rstick{\otimes} \\
  \lstick{\ket{x_1}} & \ctrl{-2} & \ \ket{x_1}\ & \ctrl{-2} & \ \ket{x_1}\ && \rstick{\ket{x_1}} \\
  \lstick{\otimes} &&&&&& \rstick{\otimes} \\
  \lstick{\ket{x_0}} &&&&& \ctrl{-4} & \rstick{\ket{x_0}}
\end{quantikz}
\end{document}
